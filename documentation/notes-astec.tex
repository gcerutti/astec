\documentclass{article}
\usepackage{fullpage}
\usepackage{amsfonts}
\usepackage[usenames,dvipsnames]{xcolor}
\usepackage[linesnumbered,boxed]{algorithm2e}
\def \mycolor {red}

\newenvironment{remarque}{\color{red}\begin{description}\item[Rq:]}
{\end{description}\color{black}}

\begin{document}

\subsection{Nomenclature}

\begin{itemize}
\item $S^{\star}_t$ segmentation de $I_{t}$, suppos\'ee correcte
\item $\tilde{S}_{t+1}$ segmentation de $I_{t+1}$ avec la segmentation projet\'ee de $I_{t}$ (les graines sont les cellules \'erod\'ees de $S^{\star}_t$ projet\'ees sur $I_{t+1}$).
\item $Seeds_{t+1}$ image des graines obtenues par s\'election \`a partir de plusieurs images de $h$-minima
\item $\hat{S}_{t+1}$ segmentation de $I_{t+1}$ \`a partir de $Seeds_{t+1}$
\end{itemize}


\subsection{\texttt{\_cell\_based\_h\_minima()}}

On calcule plusieurs images de $h$-minima r\'egionaux $Seeds^{h}_{t+1}$, avec $h \in [h_{min}, h_{max}]$. On calcule d'abord une image de minima qui contient des minima de hauteur comprise entre 1 et $h$, et on ne s\'electionne que les minima de hauteur $h$ avec un seuillage par hyst\'er\'esis. 

On ne conserve que les minima qui sont enti\`erement inclus dans une seule cellule de $\tilde{S}_{t+1}$. On \'elimine donc les graines qui chevauchent plusieurs cellules. 
\begin{remarque}
De telles graines renseigneraient pourtant sur le contraste entre 2 cellules adjacentes (sous l'hypoth\`ese que les fronti\`eres des cellules dans $\tilde{S}_{t+1}$ soient correctement localis\'ees).
\end{remarque}
\begin{remarque}
Dans \texttt{ASTEC/ASTEC.py}, le calcul de ces $h$-minima \'etait fait dans \verb|get_seeds()|, mais la v\'erification que ces graines \'etaient effectivement incluses dans les cellules de $\tilde{S}_{t+1}$ \'etait faite dans \verb|get_seeds_from_optimized_parameters()|, soit apr\`es l'appel \`a \verb|get_back_parameters()|. 
Potentiellement des graines s\'electionn\'ees dans \verb|get_back_parameters()| pouvaient donc \^etre \'elimin\'ees dans \verb|get_seeds_from_optimized_parameters()|.
\end{remarque}

\subsection{\texttt{\_select\_seed\_parameters()}}

\begin{verbatim}
    for c, s in n_seeds.iteritems():
        nb_2 = np.sum(np.array(s) == 2)
        nb_3 = np.sum(np.array(s) >= 2)
        score = nb_2*nb_3
        if (s.count(1) or s.count(2)) != 0:
            if score >= tau:
                #
                # obviously s.count(2) != 0
                # the largest h that gives 2 seeds is kept
                #
                h, sigma = parameter_seeds[c][np.where(np.array(s) == 2)[0][0]]
                nb_final = 2
            elif s.count(1) != 0:
                #
                # score < tau and s.count(1) != 0
                # the largest h that gives 1 seeds is kept
                #
                h, sigma = parameter_seeds[c][np.where(np.array(s) == 1)[0][0]]
                nb_final = 1
            else:
                #
                # score < tau and s.count(1) == 0 then obviously s.count(2)) != 0
                # the largest h that gives 1 seeds is kept
                #
                h, sigma = parameter_seeds[c][np.where(np.array(s) == 2)[0][0]]
                nb_final = 2
            selected_parameter_seeds[c] = [h, sigma, nb_final]
        #
        # s.count(1) == 0 and  s.count(2) == 0
        #
        elif s.count(3) != 0:
            h, sigma = parameter_seeds[c][s.index(3)]
            selected_parameter_seeds[c] = [h, sigma, 3]
        else:
            unseeded_cells.append(c)
            selected_parameter_seeds[c] = [0, 0, 0]
    return selected_parameter_seeds, unseeded_cells
\end{verbatim}

\color{\mycolor}
Cette fonction est l'\'equivalent de \verb|get_back_parameters()|  dans \verb|ASTEC.py|.

\begin{itemize}
\itemsep -0.5ex
\item On a $h_{min}=4$, $h_{max}=18$ et on parcourt les $h$ avec un pas  $\delta h$ de 2. L'ensemble des $h$ test\'es n'est $[h_{min}, h_{max}] \subset \mathbb{N}$ mais $\{4, 6, 8, 10, 12, 14, 16, 18\}$.
\item On calcule $NB_2$ par \verb|nb_2=np.sum(np.array(s)==2)|, c'est le nombre de $h$ qui donnent 2 graines. Elle d\'epend donc de $\delta h$
\item On calcule $NB_3$ par \verb|nb_3=np.sum(np.array(s)>=2)|, c'est le nombre de $h$ qui donnent 2 graines ou plus. Elle d\'epend aussi de $\delta h$. On a \'evidemment $NB_3 \geq NB_2$
\end{itemize}

La r\`egle impl\'ement\'ee est 
\begin{enumerate}
\itemsep -0.5ex
\item S'il existe des $h$ donnant 1 ou 2 graines (la question de la division se pose donc)
\begin{enumerate}
\itemsep -0.5ex
\item si le score $s(c) = NB_2 \cdot NB_3 \geq \tau$, alors on garde 2 graines. Comme $\tau = 25$, cela signifie que s'il y a au moins 5 valeurs de $h$ qui donnent 2 graines, alors on divisera la cellule.
\item sinon ($s(c) = NB_2 \cdot NB_3 <\tau$) et il existe des $h$ donnant 1 graine, alors on garde une graine
\item sinon ($s(c) = NB_2 \cdot NB_3 <\tau$ et il n'existe pas de $h$ donnant 1 graine) on garde 2 graines [ce cas doit difficilement survenir, il faut que beaucoup de $h$ ne donnent aucune graine, puis que les suivants donnent au moins 2 graines].
\end{enumerate}
\item sinon (il n'existe pas de $h$ donnant 1 ou 2 graines) et il existe des $h$ donnant 3 graines, alors on garde 3 graines.
\begin{remarque}
Toutefois, dans la fonction \verb|get_seeds_from_optimized_parameters()| (dans \verb|ASTEC.py| ou \verb|_build_seeds_from_selected_parameters()| dans \verb|astecnew.py|)
cr\'eant l'image des graines num\'erot\'ees \`a partir des diff\'erentes images de $h$-minima, on ne r\'ecup\`ere que les deux premi\`eres composantes num\'erot\'ees \ldots On cr\'ee donc une division, avec un choix artificiel/al\'eatoire des cellules filles.

[A CORRIGER ?! par exemple en fusionnant 2 des 3 graines]

\end{remarque}

\item sinon  (il n'existe pas de $h$ donnant 1 ou 2 ou 3 graines), on dit qu'il n'y a pas de graines
\begin{remarque}
Pourquoi ne pas consid\'erer le cas 3 graines comme le cas 4 graines ?
\end{remarque}
\end{enumerate}
On r\'ecup\`ere le premier $h$ donnant le nombre choisi de graines. Comme les $h$ sont parcourus par ordre d\'ecroissant, c'est donc le plus grand $h$ donnant ce nombre de graines qui est retenu.
\color{black}


\subsection{\texttt{\_volume\_checking()}}

\color{black}
\begin{verbatim}
[...]
    output = _volume_diagnosis(prev_volumes, curr_volumes, correspondences, parameters)
    large_volume_ratio, small_volume_ratio, small_volume_daughter, all_daughter_label = output
[...]
    seed_label_max = max(all_daughter_label)
[...]
    has_change_happened = False
    labels_to_be_fused = []
[...]
    if len(small_volume_ratio) > 0:
[...]
    for mother_c in small_volume_ratio:
[...]
        s = n_seeds[mother_c]
[...]
        nb_2 = np.sum(np.array(s) == 2)
        nb_3 = np.sum(np.array(s) >= 2)
        score = nb_2 * nb_3

        if s.count(1) > 0 or s.count(2) > 0:
            if score >= parameters.seed_selection_tau:
                nb_final = 2
            elif s.count(1) != 0:
                nb_final = 1
            else:
                nb_final = 2
[...]
            if nb_final == 1 and s.count(2) != 0:
                h_min, sigma = parameter_seeds[mother_c][s.index(2)]
                seed_image_name = common.add_suffix(membrane_image, "_seed_h" + str('{:03d}'.format(h_min)),
[...]
                bb = bounding_boxes[mother_c]
                submask_mother_c = np.zeros_like(prev_seg[bb])
                submask_mother_c[prev_seg[bb] == mother_c] = mother_c
                n_found_seeds, labeled_found_seeds = _extract_seeds(mother_c, submask_mother_c, seed_image_name, bb)
                if n_found_seeds == 2:
                    new_correspondences = [seed_label_max+1, seed_label_max+2]
[...]
                    selected_seeds_image[selected_seeds_image == correspondences[mother_c][0]] = 0
                    selected_seeds_image[bb][labeled_found_seeds == 1] = seed_label_max + 1
                    selected_seeds_image[bb][labeled_found_seeds == 2] = seed_label_max + 2
                    correspondences[mother_c] = new_correspondences
[...]
            elif (nb_final == 1 or nb_final == 2) and (np.array(s) > 2).any():
[...]
                h_min, sigma = parameter_seeds[mother_c][-1]
                seed_image_name = common.add_suffix(membrane_image, "_seed_h" + str('{:03d}'.format(h_min)),
[...]
                bb = bounding_boxes[mother_c]
                submask_daughter_c = np.zeros_like(curr_seg[bb])
                for daughter_c in correspondences[mother_c]:
                    submask_daughter_c[curr_seg[bb] == daughter_c] = mother_c
                submask_mother_c = np.zeros_like(prev_seg[bb])
                submask_mother_c[prev_seg[bb] == mother_c] = mother_c
                aux_seed_image = imread(seed_image_name)
                seeds_c = np.zeros_like(curr_seg[bb])
                seeds_c[(aux_seed_image[bb] != 0) & (submask_daughter_c == mother_c)] = 1
                seeds_c[(aux_seed_image[bb] != 0) & (submask_daughter_c == 0) & (submask_mother_c == mother_c)] = 2
[...]
                if 2 in seeds_c:
                    new_correspondences = [seed_label_max + 1, seed_label_max + 2]
[...]
                    for daughter_c in correspondences[mother_c]:
                        selected_seeds_image[selected_seeds_image == daughter_c] = 0
                    selected_seeds_image[bb][seeds_c == 1] = seed_label_max + 1
                    selected_seeds_image[bb][seeds_c == 2] = seed_label_max + 2
                    correspondences[mother_c] = new_correspondences
                    selected_parameter_seeds[mother_c] = [h_min, sigma, 2]
                    seed_label_max += 2
                    has_change_happened = True
[...]
            elif nb_final == 1:
[...]
                selected_seeds_image[selected_seeds_image == correspondences[mother_c][0]] = 0
                selected_seeds_image[deformed_seeds_image == mother_c] = correspondences[mother_c]
                selected_parameter_seeds[mother_c] = [-1, -1, 1]
                has_change_happened = True
[...]
        elif s.count(3) != 0:
[...]
            h_min, sigma = parameter_seeds[mother_c][s.index(3)]
            seed_image_name = common.add_suffix(membrane_image, "_seed_h" + str('{:03d}'.format(h_min)),
[...]
            bb = bounding_boxes[mother_c]
            submask_mother_c = np.zeros_like(prev_seg[bb])
            submask_mother_c[prev_seg[bb] == mother_c] = mother_c
            n_found_seeds, labeled_found_seeds = _extract_seeds(mother_c, submask_mother_c, seed_image_name, bb,
                                                                accept_3_seeds=True)
            if n_found_seeds == 3:
                new_correspondences = [seed_label_max + 1, seed_label_max + 2, seed_label_max + 3]
[...]
                for daughter_c in correspondences[mother_c]:
                    selected_seeds_image[selected_seeds_image == daughter_c] = 0
                selected_seeds_image[bb][labeled_found_seeds == 1] = seed_label_max + 1
                selected_seeds_image[bb][labeled_found_seeds == 2] = seed_label_max + 2
                selected_seeds_image[bb][labeled_found_seeds == 3] = seed_label_max + 3
                correspondences[mother_c] = new_correspondences
                selected_parameter_seeds[mother_c] = [h_min, sigma, n_found_seeds]
                seed_label_max += 3
                has_change_happened = True
                labels_to_be_fused.append(new_correspondences)
[...]
        else:
            monitoring.to_log_and_console('        .. (8) cell ' + str(mother_c) + ': detected seed numbers wrt h was '
                                          + str(s), 2)
[...]

    if len(large_volume_ratio) > 0:
        monitoring.to_log_and_console('      cell(s) with large increase of volume are not processed (yet)', 2)

[...]

    if len(small_volume_daughter) > 0:
        monitoring.to_log_and_console('      process cell(s) with too small daughters', 2)
        for mother_c, daughter_c in small_volume_daughter:
            selected_seeds_image[selected_seeds_image == daughter_c] = 0
            daughters_c = correspondences[mother_c]
            daughters_c.remove(daughter_c)
            if daughters_c:
                correspondences[mother_c] = daughters_c
            else:
                correspondences.pop(mother_c)
        has_change_happened = True
[...]
\end{verbatim}

\color{\mycolor}
On traite les cellules (m\`eres) ayant eu une forte diminution de volume.

La r\`egle impl\'ement\'ee est 
\begin{enumerate}
\itemsep -0.5ex
\item S'il existe des $h$ donnant 1 ou 2 graines (les tests suivants sont identiques \`a ceux de \verb|_select_seed_parameters()|)
\begin{enumerate}
\itemsep -0.5ex
\item si le score $s(c) = NB_2 \cdot NB_3 \geq \tau$, alors on garde 2 graines. Comme $\tau = 25$, cela signifie que s'il y a au moins 5 valeurs de $h$ qui donnent 2 graines, alors on divisera la cellule ($N_{daughter}=2$).
\item sinon ($s(c) = NB_2 \cdot NB_3 <\tau$) et il existe des $h$ donnant 1 graine, alors on garde une graine ($N_{daughter}=1$)
\item sinon ($s(c) = NB_2 \cdot NB_3 <\tau$ et il n'existe pas de $h$ donnant 1 graine) on garde 2 graines [ce cas doit difficilement survenir, il faut que beaucoup de $h$ ne donnent aucune graine, puis que les suivants donnent au moins 2 graines] ($N_{daughter}=2$).
\end{enumerate}

On fait ensuite les op\'erations suivantes
\begin{enumerate}
\itemsep -0.5ex
\item Si $N_{daughter}=1$ et qu'il existe un $h$ donnant 2 graines, on prend les $h$-minima pour le plus grand $h$ donnant 2 graines, et on recup\`ere ces 2 graines.
\begin{remarque} [Question pour Leo] on cr\'ee donc une division l\`a o \`u on n'en voulait pas ?!
\end{remarque}
\item sinon si $N_{daughter}=1$ ou $N_{daughter}=2$ et qu'il existe un $h$ donnant plus de 2 graines, on prend les $h$-minima pour le plus petit $h$ (c-\`a-d $h_{min}$).
Les graines dans $\tilde{S}_{t+1}(mother) \cap \bigcup S(daugther)$, c-\`a-d les graines \`a la fois dans la segmentation construite avec les \'erod\'es des cellules m\`eres et la segmentation courante sont \'etiquet\'ees 1, tandis que celles uniquement dans $\tilde{S}_{t+1}(mother)$ (donc pas dans la segmentation courante $\bigcup S(daugther)$ sont \'etiquet\'ees 2. Potentiellement, ces graines \`a '2' permettent de r\'ecup\'erer de la mati\`ere pour les filles, donc de r\'eduire la diminution de volumes.
S'il y a effectivement des '2',
avec ces '1' et ces '2', on cr\'ee 2 graines pour la segmentation, donc on force $N_{daughter}=2$.
\begin{remarque} [Question pour Leo] 1) on cr\'ee donc une division l\`a o\`u on n'en voulait pas ?! 2) Pourquoi ne pas avoir v\'erifi\'e, dans le cas pr\'ec\'edent, qu'il y avait bien des nouvelles graines donnant potentiellement plus de mati\`ere ?!
\end{remarque}
\item sinon si $N_{daughter}=1$ (il n'y a pas de $h$ donnant plus de une graine), on r\'ecup\`ere la graine \'erod\'ee de $\tilde{S}_{t+1}(mother)$
\end{enumerate}

\item sinon  (il n'existe pas de $h$ donnant 1 ou 2 ou 3 graines). Cela peut arriver si la cellule fille avait d\'ej\`a \'et\'e cr\'e\'ee gr\^ace \`a l'\'erod\'e de la cellule m\`ere,

\end{enumerate}


\color{black}


\subsection{\texttt{astec\_process()}}
\color{black}
\begin{verbatim}
[...]
    membrane_image = reconstruction.build_membrane_image(current_time, experiment, parameters,
                                                         previous_time=previous_time)
[...]
\end{verbatim}
\color{\mycolor}
Reconstruction d'une image de membrane 
(\verb|reconstruction.build_membrane_image()|), typiquement appel au rehaussement de membrane et fusion avec l'image originale.

\color{black}
\begin{verbatim}
[...]
    previous_segmentation = experiment.get_segmentation_image(previous_time)
[...]
    undeformed_seeds = common.add_suffix(previous_segmentation, '_undeformed_seeds_from_previous',
                                         new_dirname=experiment.astec_dir.get_tmp_directory(),
                                         new_extension=experiment.default_image_suffix)
    _build_seeds_from_previous_segmentation(previous_segmentation, undeformed_seeds, parameters)
    deformed_seeds = common.add_suffix(previous_segmentation, '_deformed_seeds_from_previous',
                                       new_dirname=experiment.astec_dir.get_tmp_directory(),
                                       new_extension=experiment.default_image_suffix)
    deformation = reconstruction.get_deformation_from_current_to_previous(current_time, experiment,
                                                                          parameters, previous_time)
    cpp_wrapping.apply_transformation(undeformed_seeds, deformed_seeds, deformation,
                                      interpolation_mode='nearest', monitoring=monitoring)
[...]
    if parameters.propagation_strategy is 'seeds_from_previous_segmentation':
        segmentation_from_previous = astec_image
    else:
        segmentation_from_previous = common.add_suffix(membrane_image, '_watershed_from_previous',
                                                       new_dirname=experiment.astec_dir.get_tmp_directory(),
                                                       new_extension=experiment.default_image_suffix)
    mars.watershed(deformed_seeds, membrane_image, segmentation_from_previous, experiment, parameters)
[...]
\end{verbatim}
\color{\mycolor}
Calcul de $\tilde{S}_{t+1}$, c'est-\`a-dire une estimation de la segmentation de $I_{t+1}$ \` partir de celle de $I_{t}$, sans division cellulaire
\begin{enumerate}
\itemsep -0.5ex
\item Les cellules de $S^{\star}_{t}$ sont \'erod\'ees pour former l'image de graines $S^{e}_{t}$ (\verb|undeformed_seeds|)
\begin{remarque}
L'\'erosion se fait en 26-connexit\'e, avec un nombre d'it\'erations diff\'erent pour les cellules (10) et le fond (25)). 
\end{remarque}
\item Cette image de graines est transform\'ee vers $I_{t+1}$
pour donner $S^{e}_{t+1 \leftarrow t}$ (\verb|deformed_seeds|)
\begin{displaymath}
S^{e}_{t+1 \leftarrow t} = S^{e}_{t} \circ \mathcal{T}_{t \leftarrow t+1}
\end{displaymath}
\item On r\'ealise une ligne de partage des eaux pour obtenir l'image $\tilde{S}_{t+1}$ (\verb|segmentation_from_previous|). 
\end{enumerate}

\color{black}
\begin{verbatim}
[...]
    n_seeds, parameter_seeds = _cell_based_h_minima(segmentation_from_previous, cells, bounding_boxes, membrane_image,
                                                    experiment, parameters)
[...]
    selected_parameter_seeds, unseeded_cells = _select_seed_parameters(n_seeds, parameter_seeds,
[...]
    selected_seeds = common.add_suffix(membrane_image, '_seeds_from_selection',
                                       new_dirname=experiment.astec_dir.get_tmp_directory(),
                                       new_extension=experiment.default_image_suffix)

    output = _build_seeds_from_selected_parameters(selected_parameter_seeds, segmentation_from_previous, deformed_seeds,
                                                   selected_seeds, cells, unseeded_cells, bounding_boxes,
                                                   membrane_image, experiment, parameters)

    label_max, correspondences, divided_cells = output
[...]
    segmentation_from_selection = common.add_suffix(membrane_image, '_watershed_from_selection',
                                                    new_dirname=experiment.astec_dir.get_tmp_directory(),
                                                    new_extension=experiment.default_image_suffix)
    mars.watershed(selected_seeds, membrane_image, segmentation_from_selection, experiment, parameters)
\end{verbatim}
\color{\mycolor}
\begin{enumerate}
\itemsep -0.5ex
\item \verb|_cell_based_h_minima()| calcule, pour chaque cellule de $\tilde{S}_{t+1}$, le nombre de $h$-minima (les graines) pour un ensemble de valeurs de $h$.

\item \verb|_select_seed_parameters()| calcule, pour chaque cellule de $\tilde{S}_{t+1}$, le nombre de graines qui sera retenu (et donc un $h$ optimal). Ce nombre peut varier de 0 \`a 3. 

\item \verb|_build_seeds_from_selected_parameters()| cr\'ee l'image des graines $Seeds_{t+1}$ \`a partir des $h$ s\'electionn\'es auparavant. Ajoute aussi une graine pour le fond.

\item \verb|mars.watershed()| r\'ealise une segmentation de $I_{t+1}$ \`a partir de $Seeds_{t+1}$. Cette segmentation est not\'ee $\hat{S}_{t+1}$.

\end{enumerate}

\color{black}













\pagebreak





\begin{itemize}

\item reconstruction d'une image de membrane 
(\verb|reconstruction.build_membrane_image()|)

\item calcul de $\tilde{S}_{t+1}$, c'est-\`a-dire la segmentation $S^{\star}_{t}$ d\'eform\'ee \`a $t+1$ 
\begin{displaymath}
\tilde{S}_{t+1} = S^{\star}_{t} \circ \mathcal{T}_{t \leftarrow t+1}
\end{displaymath}

\item calcul des images de graines $\mathrm{Seeds}^{h}_{t+1}$ pour $h \in [h_{min}, h_{max}]$ (\verb|_cell_based_h_minima()|)

\item s\'election des param\`etres, ie du nombre de graines pour chaque cellule "m\`ere" (\verb|_select_seed_parameters()|)

\end{itemize}





\section{Segmentation de l'image \`a $t+1$}

\subsection{$h$-minima}

On calcule plusieurs images de $h$-minima r\'egionaux $Seeds^{h}_{t+1}$, avec $h \in [h_{min}, h_{max}]$. On calcule d'abord une image de minima qui contient des minima de hauteur comprise entre 1 et $h$, et on ne s\'electionne que les minima de hauteur $h$ avec un seuillage par hyst\'er\'esis. 

On ne conserve que les minima qui sont enti\`erement inclus dans une seule cellule de $\tilde{S}_{t+1}$. On \'elimine donc les graines qui chevauchent plusieurs cellules. 
\begin{remarque}
De telles graines renseigneraient pourtant sur le contraste entre 2 cellules adjacentes (sous l'hypoth\`ese que les fronti\`eres des cellules dans $\tilde{S}_{t+1}$ soient correctement localis\'ees).
\end{remarque}




\subsection{\texttt{get\_seeds\_from\_optimized\_parameters()}}

La fonction \verb|get_seeds_from_optimized_parameters()| (dans \verb|ASTEC.py| ou \verb|_build_seeds_from_selected_parameters()| dans \verb|astecnew.py|) construit une image de graines \`a partir des param\`etres retenus dans \verb|get_back_parameters()|.

\begin{itemize}
\item 1 graine: on r\'ecup\`ere la graine dans $Seeds^{h}_{t+1}$ pour la cellule $c$

\item 2 graines: on r\'ecup\`ere les 2 graines dans  $Seeds^{h}_{t+1}$ pour la cellule $c$

\item 3 graines: on ne r\'ecup\`ere que 2 graines (les deux premi\`eres num\'erot\'ees) dans  $Seeds^{h}_{t+1}$ pour la cellule $c$.
\begin{remarque}
Ce comportement doit \^etre corrig\'e.
\end{remarque}

\item 0 graine: on regarde si le volume de la cellule $c$ (dans $\tilde{S}_{t+1}$) est suffisamment grand (sup\'erieur \`a 100). 
Si oui, on r\'ecup\`ere alors la graine correspondante dans $S^e_{t+1 \leftarrow t} = S^e_t \circ \mathcal{T}_{t \leftarrow t+1}$ (cellules de $S^{\star}_t$ \'erod\'ees (10 it\'erations en 26-connexit\'e) puis transform\'ees).
\begin{remarque}
Il y a ici une potentielle fin de lin\'eage.
\end{remarque}

\item fond: on r\'ecup\`ere toutes les graines de $Seeds^{h_{min}}_{t+1}$ qui correspondent \`a la cellule $1$ (le fond). 

\end{itemize}

On 


\subsection{\texttt{volume\_checking()}}

Cherche \`a corriger des erreurs d\'etect\'ees par des changements de volume trop importants.

Pour une cellule $c^{t}\_i$


\cite{guignard:tel-01278725}

\bibliographystyle{unsrt}
\bibliography{bib-astec}

\end{document}
