\section{\texttt{5-postcorrection.py}}
\label{sec:cli:post:correction}

\subsection{Post-correction overview}

The Astec segmentation procedure yields a series of segmented images $\{S^{\star}_t\}_t$, where each segmented image $S^{\star}_{t+1}$ takes advantage of the knowledge of the previously segmented image $S^{\star}_t$ to decide, at the cell level, whether a cell division may occur. However, there are still segmentation errors, that can be detected from the study of the overall lineage (see \cite[section 2.3.3.7, page 74]{guignard:tel-01278725} and \cite[supp. mat.]{guignard:hal-02903409}).

As suggested by its name, the post-correction will try to a posteriori correct the segmentation resulting from the  \verb|4-astec.py| stage (see section \ref{sec:cli:astec}). 

The post-correction is made of the following steps.
\begin{enumerate}
\item \label{it:post:pruning} Lineage pruning: 
it goes through the end branches (a branch does not have any cell division; an end branch finishes either at the end of the sequence or the cell vanishes between two time points) of the lineage tree.
Some lineage end branches are deleted (see section \ref{sec:cli:post:correction:pruning} for details), meaning that the corresponding cells are fused with other cells of the embryo.
\item \label{it:post:division} Division postponing: some divisions are postponed.
\end{enumerate}



\subsection{\texttt{5-postcorrection.py} options}

The following options are available:
\begin{description}
  \itemsep -0.5ex
\item[\texttt{-h}] prints a help message
\item[\texttt{-p \underline{file}}] set the parameter file to be parsed
\item[\texttt{-e \underline{path}}] set the
  \texttt{\underline{path}} to the directory where the
  \texttt{RAWDATA/} directory is located
\item[\texttt{-k}] allows to keep the temporary files
\item[\texttt{-f}] forces execution, even if (temporary) result files
  are already existing
\item[\texttt{-v}] increases verboseness (both at console and in the
  log file)
\item[\texttt{-nv}] no verboseness
\item[\texttt{-d}]  increases debug information (in the
  log file)
\item[\texttt{-nd}] no debug information
\end{description}

\subsection{Input data}
\label{sec:cli:post:correction:input:data}
Input data are the result of the \verb|4-astec.py| stage (see section \ref{sec:cli:astec}) and will be searched in the directory \verb|SEG/SEG\_<EXP\_SEG>/| (see section \ref{sec:cli:astec:output:data}).

\dirtree{%
.1 /path/to/experiment/.
.2 \ldots.
.2 SEG/.
.3 SEG\_<EXP\_SEG>/.
.4 <EN>\_seg\_lineage.xml.
.4 <EN>\_seg\_t<begin>.mha.
.4 <EN>\_seg\_t<..>.mha.
.4 <EN>\_seg\_t<end>.mha.
.4 \ldots.
.2 \ldots.
}


\subsection{Output data}

The results are stored in sub-directories
\texttt{POST/POST\_<EXP\_POST>} under the
\texttt{/path/to/experiment/} directory where where \texttt{<EXP\_POST>} is the value of the variable \texttt{EXP\_POST} (its
default value is '\texttt{RELEASE}'). 

\dirtree{%
.1 /path/to/experiment/.
.2 \ldots.
.2 POST/.
.3 POST\_<EXP\_POST>/.
.4 <EN>\_post\_lineage.xml.
.4 <EN>\_post\_t<begin>.mha.
.4 <EN>\_post\_t<..>.mha.
.4 <EN>\_post\_t<end>.mha.
.4 LOGS/.
.2 \ldots.
}

The image format to be used (here \verb|mha|) is given by the variable \texttt{result\_image\_suffix}, while the lineage format to be used (here \verb|xml|) is given by the variable \texttt{result\_lineage\_suffix}.





\subsection{Step \ref{it:post:pruning}: lineage pruning}
\label{sec:cli:post:correction:pruning}

Bifurcations of the lineage tree correspond to cell division, while branches (between two bifurcations or between a bifurcation and a leaf) corresponds to the lifespan of a cell.
The purpose of this step is to detect suspicious end branches (terminating by a leaf) that may correspond to an over-segmentation error. 

An end branch is candidate for deletion if
\begin{itemize}
\itemsep -0.5ex
\item either it terminates before the last time point (it corresponds then to a cell without daughter cell in the next time point), 
\item or the volume of its last cell is too small (threshold given by the variable \texttt{postcorrection\_volume\_minimal\_value}).
\end{itemize}

An end branch candidate for deletion is deleted if
\begin{itemize}
\itemsep -0.5ex
\item either it is too short (threshold given by the variable \texttt{postcorrection\_lifespan\_minimal\_value}),
\item or either its sister branch (which may not be an end branch) or its mother branch is too short, meaning that there are two divisions too close, (thresholds still given by the variable \texttt{postcorrection\_lifespan\_minimal\_value}),
\item or (if the variable \texttt{postcorrection\_test\_early\_division} is set to \texttt{True}) if the Pearson correlation coefficient between the volumes of the candidate end branch and its sister branch is less than 
-\texttt{postcorrection\_correlation\_threshold}, meaning that the volumes are anti-correlated (typically the volumes of the candidate end branch are decreasing while the ones of the sister branch are increasing, indicating a fake division detection).
\end{itemize}

\subsection{Step \ref{it:post:division}: division postponing}

      
\begin{itemize}
\itemsep -0.5ex
\item \texttt{postcorrection\_volume\_minimal\_value}
  branch ending with leaf cell below this value are candidate for deletion. Expressed in voxel unit.
\item \texttt{postcorrection\_lifespan\_minimal\_value}
\item \texttt{postcorrection\_test\_early\_division}
\item \texttt{postcorrection\_test\_volume\_correlation}
\item \texttt{postcorrection\_correlation\_threshold}
\item \texttt{postcorrection\_lineage\_diagnosis}
  performs a kind of diagnosis on the lineage before and after the post-correction.
\end{itemize}






