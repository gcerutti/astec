\section{\texttt{5-postcorrection.py}}
\label{sec:cli:post:correction}

\subsection{\texttt{5-postcorrection.py} options}

The following options are available:
\begin{description}
  \itemsep -0.5ex
\item[\texttt{-h}] prints a help message
\item[\texttt{-p \underline{file}}] set the parameter file to be parsed
\item[\texttt{-e \underline{path}}] set the
  \texttt{\underline{path}} to the directory where the
  \texttt{RAWDATA/} directory is located
\item[\texttt{-k}] allows to keep the temporary files
\item[\texttt{-f}] forces execution, even if (temporary) result files
  are already existing
\item[\texttt{-v}] increases verboseness (both at console and in the
  log file)
\item[\texttt{-nv}] no verboseness
\item[\texttt{-d}]  increases debug information (in the
  log file)
\item[\texttt{-nd}] no debug information
\end{description}

\subsection{Output data}

The results are stored in sub-directories
\texttt{POST/POST\_<EXP\_POST>} under the
\texttt{/path/to/experiment/} directory where where \texttt{<EXP\_POST>} is the value of the variable \texttt{EXP\_POST} (its
default value is '\texttt{RELEASE}'). 

\dirtree{%
.1 /path/to/experiment/.
.2 \ldots.
.2 POST/.
.3 POST\_<EXP\_POST>/.
.4 <EN>\_post\_lineage.xml.
.4 <EN>\_post\_t<begin>.mha.
.4 <EN>\_post\_t<..>.mha.
.4 <EN>\_post\_t<end>.mha.
.4 LOGS/.
.2 \ldots.
}

The image format to be used (here \verb|mha|) is given by the variable \texttt{result\_image\_suffix}, while the lineage format to be used (here \verb|xml|) is given by the variable \texttt{result\_lineage\_suffix}.

\subsection{Post-correction parameters}

As suggested by its name, the post-correction will try to a posteriori correct the segmentation resulting from the  \verb|4-astec.py| stage (see section \ref{sec:cli:astec}). It will goes through the end branches (a branch does not have any cell division; an end branch finishes either at the end of the sequence or the cell vanishes between two time points) of the lineage tree.

\subsubsection{Input data}
\label{sec:cli:post:correction:input:data}
Input data are the result of the \verb|4-astec.py| stage (see section \ref{sec:cli:astec}) and will be searched in the directory \verb|SEG/SEG\_<EXP\_SEG>/| (see section \ref{sec:cli:astec:output:data}).

\subsubsection{En branches}
\label{sec:cli:post:correction:end:branches}

An end branch is candidate for deletion (ie fusion with its sister branch) if
\begin{itemize}
\itemsep -0.5ex
\item it finishes before the end of the sequence, or
\item the volume of the leaf cell is less than the value of the variable \texttt{volume\_minimal\_value}
\end{itemize}

An end branch will be deleted (ie fused with its sister branch) if
\begin{itemize}
\itemsep -0.5ex
\item its length is (strictly) less than the value of the variable \texttt{lifespan\_minimal\_value}, or
\item \textbf{to be fixed} if the variable  \texttt{test\_early\_division} is set to \texttt{True}, if the length if its sister branch (until a leaf or a division) is less than \texttt{lifespan\_minimal\_value}, meaning that there is two too close cell divisions, or
\item if the variable \texttt{test\_volume\_correlation} is set to \texttt{True}, if the Pearson correlation coefficient between the volumes of the candidate end branch and its sister branch is less than 
-\texttt{correlation\_threshold}, meaning that the volumes are anti-correlated (typically the volumes of the candidate end branch are decreasing while the ones of the sister branch are increasing, indicating a fake division detection).
\end{itemize}



\subsection{Parameter list}

Please also refer to the file
\texttt{parameter-file-examples/5-postcorrection-parameters.py}

\begin{itemize}
\itemsep -0.5ex
\item \texttt{EN}
\item \texttt{EXP\_POST}
\item \texttt{EXP\_SEG}
      defines the segmentation directory where to find the input data (see section \ref{sec:cli:post:correction:input:data}).
\item \texttt{PATH\_EMBRYO}
\item \texttt{begin}
\item \texttt{correlation\_threshold}
\item \texttt{end}
\item \texttt{lifespan\_minimal\_value}
\item \texttt{lineage\_diagnosis}
      performs a kind of diagnosis on the lineage before and after the post-correction.
\item \texttt{postcor\_PearsonThreshold} 
      same as \texttt{correlation\_threshold}
\item \texttt{postcor\_Soon}
      same as \texttt{test\_early\_division}
\item \texttt{postcor\_Volume\_Threshold} 
      same as \texttt{volume\_minimal\_value}
\item \texttt{postcor\_correlation\_threshold}
\item \texttt{postcor\_lifespan\_minimal\_value}
      same as \texttt{lifespan\_minimal\_value}
\item \texttt{postcor\_lineage\_diagnosis}
      same as \texttt{ineage\_diagnosis}
\item \texttt{postcor\_pearson\_threshold} 
      same as \texttt{correlation\_threshold}
\item \texttt{postcorrection\_correlation\_threshold}
       same as \texttt{correlation\_threshold}
\item \texttt{postcorrection\_lifespan\_minimal\_value}
      same as \texttt{lifespan\_minimal\_value}
\item \texttt{postcorrection\_lineage\_diagnosis}
      same as \texttt{ineage\_diagnosis}
\item \texttt{postcorrection\_pearson\_threshold}
      same as \texttt{correlation\_threshold}
\item \texttt{postcorrection\_soon}
      same as \texttt{test\_early\_division}
\item \texttt{postcorrection\_test\_early\_division}
      same as \texttt{test\_early\_division}
\item \texttt{postcorrection\_test\_volume\_correlation}
     same as \texttt{test\_volume\_correlation}
\item \texttt{postcorrection\_volume\_minimal\_value}
      same as \texttt{volume\_minimal\_value}
\item \texttt{result\_image\_suffix}
\item \texttt{result\_lineage\_suffix}
\item \texttt{test\_early\_division}
\item \texttt{test\_volume\_correlation}
\item \texttt{volume\_minimal\_value}
      branch ending with leaf cell below this value are candidate for deletion. Expressed in voxel unit.
\end{itemize}
